\documentclass[12pt]{article}
\usepackage{polski}
\usepackage[utf8]{inputenc}
\usepackage[T1]{fontenc}
\usepackage{amsmath}
\usepackage{amsfonts}
\usepackage{fancyhdr}
\usepackage{lastpage}
\usepackage{multirow}
\usepackage{amssymb}
\usepackage{amsthm}
\usepackage{textcomp}
\frenchspacing
\usepackage{fullpage}
\setlength{\headsep}{30pt}
\setlength{\headheight}{12pt}
%\setlength{\voffset}{-30pt}
%\setlength{\textheight}{730pt}
\pagestyle{myheadings}
%\usepackage{kuvio,amscd,diagrams,dcpic,xymatrix,diagxy}
\usepackage{tikz,paralist,mathtools}
\usetikzlibrary{matrix,arrows,decorations}

\DeclareMathOperator{\coker}{coker}
\DeclareMathOperator{\Hom}{Hom}
\DeclareMathOperator{\Tor}{Tor}
\DeclareMathOperator{\Mor}{Mor}
\DeclareMathOperator{\Kom}{Kom}
\DeclareMathOperator{\sgn}{sgn}
\DeclareMathOperator{\Ext}{Ext}
\DeclareMathOperator{\Ob}{Ob}
\DeclareMathOperator{\Cone}{C}
\DeclareMathOperator{\Cyl}{Cyl}
\DeclareMathOperator{\Der}{D}
\DeclareMathOperator{\K}{K}
\newcommand{\Set}{\mathrm{Set}}
\newcommand{\Top}{\mathrm{Top}}
\newcommand{\Rmod}{\mathrm{R-mod}}
\newcommand{\id}{\mathrm{id}}
\newcommand{\A}{\mathcal{A}}
\newcommand{\B}{\mathcal{B}}
\newcommand{\C}{\mathcal{C}}
\newcommand{\D}{\mathcal{D}}
\newcommand{\Q}{\mathcal{Q}}
\newcommand{\qi}{quasi-isomorphism}

\newcommand{\bigslant}[2]{{\left.\raisebox{.2em}{$#1$}\middle/\raisebox{-.2em}{$#2$}\right.}}
\newcommand{\mf}[1]{{\mathfrak{#1}}}
\newcommand{\mb}[1]{{\mathbb{#1}}}
\newcommand{\mc}[1]{{\mathcal{#1}}}
\newcommand{\mr}[1]{{\mathrm{#1}}}


\newcounter{punkt}

\theoremstyle{plain}
\newtheorem{theorem}[punkt]{Theorem}
\newtheorem{lemma}[punkt]{Lemma}
\newtheorem{definition}[punkt]{Definition}
\newtheorem{fact}[punkt]{Fact}
\newtheorem{proposition}[punkt]{Proposition}
\newtheorem{corollary}[punkt]{Corollary}

\theoremstyle{definition}
\newtheorem{remark}[punkt]{Remark}
\newtheorem{example}[punkt]{Example}
\newtheorem{exercise}[punkt]{Exercise}



\markright{Piotr Suwara\hfill Homological algebra II: January 14, 2014\hfill}

\begin{document}
    {\bf Applications of stable derived functors}
    
    \begin{theorem}
        $T:\Rmod \to \Rmod$, then 
        $$L_i^sT(A) = \lim_n \pi_{i+n}(T(\tilde{R}[S^n] \otimes P_\ast))
        \lim_n H_{i+n}(T(\tilde{R}[S^n] \otimes P_\ast)),$$
        where $S^n$ is any simplicial model of $n$-sphere,
        $\tilde{R}[\gamma] = \bigslant{R[\gamma]}{R[\ast]}$ a~simplicial set,
        $P_\ast$ is any projective resolution of $A$.
        
        The limit is taken via suspension
        \\ $\pi_{i+n}(T(\tilde{R}[S^n] \otimes P_\ast)) 
        \to \pi_{i+n+1}(S^1 \wedge T(\tilde{R}[S^n] \otimes P_\ast))
        \to \pi_{i+n+1}(T(\tilde{R}[S^{n+1}] \otimes P_\ast))$.
    \end{theorem}
        
    In general for $S^1 \wedge F(X) \to F(S^1 \wedge X)$
    one has to have for any $z \in S^1$,
    ${F(X) \to F(S^1 \wedge X)}$, $X \to S^1 \wedge X$, $x \to z \wedge x$.
    
    One takes $R = \mb Z / p$ or $R = \mb Z$.
    
    $L_i^sT(\mb Z/p) = \lim \pi_{i+n} T(\mb Z/p [S^n])$,
    but $\widetilde{\mb Z/p}[S^n] = K(\mb Z/p,n)$,
    $\tilde{\mb Z}[S^n] = K(\mb Z, n)$.
    
    Stalk skewed gra..itions on $H^\ast(\bullet, \mb Z/p)$
    is \\$H_\ast^s(K(\mb Z/p),\mb Z/p) 
    = H_\ast^s(K(\mb Z/p,n),\mb Z/p)
    = L_\ast^s \mb Z_p[.](\mb Z/p)$. (?)
    
    \begin{theorem}
        Let $SP^i$ be the $i$-th symmetric power functor,
        and $SP^i_p$ the $p$-reduced $i$-th symmetric power,
        and $SP^\ast_p = \bigslant{\bigoplus SP^i}{\langle x^p - 1 \rangle}$.
        
        Then $L_\ast^s SP^\ast(\mb Z/p) = H_\ast^s(K(\mb Z),\mb Z/p)$,
        $L_\ast^s SP_p^\ast(\mb Z/p) = H_\ast^s(K(\mb Z/p),\mb Z/p)$.
    \end{theorem}
    
    Calculations: Let $\Gamma$ be a~category of functors 
    $T:$ finite pointed sets $\to \mb Z/p\mr{-vect}$,
    \\$T(\ast)=0$. 
    $L \in \Gamma$ is defined as $L(X) = \widetilde{\mb Z/p}[X]$.
    
    \begin{lemma}
        Let $T:\mb Z/p\mr{-vect} \to \mb Z/p\mr{-vect}$.
        Then $L_i^sT(\mb Z/p) = \Tor_i^\Gamma(L^\ast,T\circ L)$,
        where $L^\ast(X) = L(X)^\ast$, and
    \end{lemma}
    
    \sout{OK, I am blown up. Break.}
    
    I have found 
    \href{http://www-irma.u-strasbg.fr/~vespa/Cesaro.pdf}{these notes}
    useful in understanding derived functors.
    
    Some detailed constructions are 
    \href{http://www.math.ku.dk/english/research/top/paststudents/gabe.msproject.2010.pdf}
    {here},
    and some worked out examples are
    \href{http://arxiv.org/pdf/0910.2817.pdf}{here} in section 2.2.



\end{document}
 
 