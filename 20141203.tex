\documentclass[12pt]{article}
\usepackage{polski}
\usepackage[utf8]{inputenc}
\usepackage[T1]{fontenc}
\usepackage{amsmath}
\usepackage{amsfonts}
\usepackage{fancyhdr}
\usepackage{lastpage}
\usepackage{multirow}
\usepackage{amssymb}
\usepackage{amsthm}
\usepackage{textcomp}
\frenchspacing
\usepackage{fullpage}
\setlength{\headsep}{30pt}
\setlength{\headheight}{12pt}
%\setlength{\voffset}{-30pt}
%\setlength{\textheight}{730pt}
\pagestyle{myheadings}
%\usepackage{kuvio,amscd,diagrams,dcpic,xymatrix,diagxy}
\usepackage{tikz,paralist,mathtools}
\usetikzlibrary{matrix,arrows,decorations}

\DeclareMathOperator{\coker}{coker}
\DeclareMathOperator{\Hom}{Hom}
\DeclareMathOperator{\Tor}{Tor}
\DeclareMathOperator{\Mor}{Mor}
\DeclareMathOperator{\Kom}{Kom}
\DeclareMathOperator{\sgn}{sgn}
\DeclareMathOperator{\Ext}{Ext}
\DeclareMathOperator{\Ob}{Ob}
\DeclareMathOperator{\Cone}{C}
\DeclareMathOperator{\Cyl}{Cyl}
\DeclareMathOperator{\Der}{D}
\DeclareMathOperator{\K}{K}
\newcommand{\Set}{\mathrm{Set}}
\newcommand{\Top}{\mathrm{Top}}
\newcommand{\Rmod}{\mathrm{R-mod}}
\newcommand{\id}{\mathrm{id}}
\newcommand{\A}{\mathcal{A}}
\newcommand{\B}{\mathcal{B}}
\newcommand{\C}{\mathcal{C}}
\newcommand{\D}{\mathcal{D}}
\newcommand{\Q}{\mathcal{Q}}
\newcommand{\qi}{quasi-isomorphism}

\newcommand{\bigslant}[2]{{\left.\raisebox{.2em}{$#1$}\middle/\raisebox{-.2em}{$#2$}\right.}}
\newcommand{\mf}[1]{{\mathfrak{#1}}}
\newcommand{\mb}[1]{{\mathbb{#1}}}
\newcommand{\mc}[1]{{\mathcal{#1}}}
\newcommand{\mr}[1]{{\mathrm{#1}}}


\newcounter{punkt}

\theoremstyle{plain}
\newtheorem{theorem}[punkt]{Theorem}
\newtheorem{lemma}[punkt]{Lemma}
\newtheorem{definition}[punkt]{Definition}
\newtheorem{fact}[punkt]{Fact}
\newtheorem{proposition}[punkt]{Proposition}
\newtheorem{corollary}[punkt]{Corollary}

\theoremstyle{definition}
\newtheorem{remark}[punkt]{Remark}
\newtheorem{example}[punkt]{Example}
\newtheorem{exercise}[punkt]{Exercise}



\markright{Piotr Suwara\hfill Homological algebra II: December 3, 2014\hfill}

\begin{document}
    Let $\C$ be abelian,
    remind $s\C$ -- simplicial objects in $\C$,
    $C_\ast(\C)$ -- chain complexes over $\C$.
    
    \begin{definition}[$s$-morphism]
        $X_\bullet, Y_\bullet \in s\C$. For a~simplicial set $K \in s\Set$
        a~map which associates $F(\sigma):X_n \to Y_n$
        to any $\sigma \in K_n$ is called {\em $s$-morphism}
        (denote $F:K \times X_\bullet \to Y_\bullet$)
        if for any $\alpha:[m]\to[n]$ in $\Delta$ we have
        $F(K(\alpha)(\sigma)) \circ X(\alpha) = Y(\alpha) F(\sigma)$.
        
        Observe that when $X_\bullet, Y_\bullet$ are in $s\Set$ then
        $s$-morphisms are simplicial maps 
        $K_\bullet \times X_\bullet \to Y_\bullet$.
    \end{definition}
    
    \begin{example}
        If $K=\Delta[0]$, then $s$-morphism 
        is just a~simplicial morphism $X_\bullet \to Y_\bullet$.
    \end{example}
    
    \begin{example}
        If $K=\Delta[1]$ then $s$-morphism is called
        {\em a homotopy} between $F(0)$ and $F(1)$.
    \end{example}
    
    \begin{remark}
        $T:\C \to \C'$ functor
        induces $T:s\C \to s\C'$, 
        if $X_\bullet \in s\C$, then
        $T(X)_n = T(X_n), T(X)(\alpha) = T(X(\alpha))$,
    \end{remark}
    
    \begin{definition}
        If $F$ is an $s$-morphism 
        $F:K_\bullet \times X_\bullet \to Y_\bullet$,
        then it defines 
        $TF:K_\bullet \times T(X_\bullet) \to T(Y_\bullet)$
        defined by $TF(\sigma) = T(F(\sigma))$.
    \end{definition}
    
    \begin{remark}
        Any functor $T:\C \to \C'$ sends homotopic maps
        to homotopic ones.
    \end{remark}

\pagebreak    
    \begin{definition}
        Let $\C$ be abelian, then there are functors 
        $$ s\C \xrightleftharpoons[K]{N} C_\ast(\C)$$
        defined as follows.
        
        Normalization $N$ is defined, for $X_\bullet \in s\C$, as
        $$N(X)_n = \bigcap_{i=1}^n \ker(d_i:X_n \to X_{n-1})$$
        (e.g. $\ker (X_n \xrightarrow{\prod d_i} \prod X_{n-1}$),
        with $$d:N(X)_n \to N(X)_{n-1}$$
        induced by $d_0$.
        
        $K$ is defined in such a~way. 
        If $\alpha:[n]\to[q]$, then $d(\alpha) = n$ and $r(\alpha)=q$.
        Notice for any $\alpha$ there is unique
        $\alpha=\varepsilon \circ \eta$,
        where $\varepsilon$ is an injection and $\eta$ is a~surjection.
        For $C \in C_\ast(\C)$, take
        $$K(C)_n = \bigoplus_{\eta:d(\eta)=n} C_{r(\eta)},$$
        Now for $\alpha :[m]\to[n]$
        define $$KC(\alpha):K(C)_n \to K(C)_m$$
        on every $C_{r(\eta)}$ in such a~way:
        $\eta \alpha = \varepsilon' \eta'$,
        let $KC(\alpha)$ map $C_{r(\eta)}$
        into $C_{r(\eta')}$
        via the formula
        $$K(\eta, \alpha) = 
        \begin{cases}
            \id_{C_{r(\eta)}} &\text{for }\varepsilon'=\id_{[r(\eta)]}\\
            d:C_{r(\eta)}\to C_{r(\eta)-1} = C_{r(\eta')} 
                &\text{for }\varepsilon'=\varepsilon^0 \\
            0 &\text{otherwise}
        \end{cases}.$$
    \end{definition}
    
    \begin{remark}
        Observe that if $f:C\to D$ in $C_\ast(\C)$,
        then the induced map $KC \to KD$ is simplicial.
    \end{remark}
    
    \begin{theorem}[Dold-Kan]
        The functors $N$ and $K$ give 
        an equivalence of $s\C$ and $C_\ast(\C)$.
    \end{theorem}
    
    \begin{remark}
        It was somehow convenient to define,
        for $X \in s\C$, $\bar{X} \in s\C$
        via $\bar{X}_n = \ker(d_{n+1}:X_{n+1} \to X_n)$.
    \end{remark}
    
    \begin{lemma}
        Let $f:X_\bullet \to Y_\bullet$ be a~simplicial morphism 
        which satisfies $(Nf)_i$ is mono(epi) for $i \leq n$.
        Then $f_n:X_n \to Y_n$ is mono(epi) for $i \leq n$.
    \end{lemma}


\end{document}
 
 