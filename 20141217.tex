\documentclass[12pt]{article}
\usepackage{polski}
\usepackage[utf8]{inputenc}
\usepackage[T1]{fontenc}
\usepackage{amsmath}
\usepackage{amsfonts}
\usepackage{fancyhdr}
\usepackage{lastpage}
\usepackage{multirow}
\usepackage{amssymb}
\usepackage{amsthm}
\usepackage{textcomp}
\frenchspacing
\usepackage{fullpage}
\setlength{\headsep}{30pt}
\setlength{\headheight}{12pt}
%\setlength{\voffset}{-30pt}
%\setlength{\textheight}{730pt}
\pagestyle{myheadings}
%\usepackage{kuvio,amscd,diagrams,dcpic,xymatrix,diagxy}
\usepackage{tikz,paralist,mathtools}
\usetikzlibrary{matrix,arrows,decorations}

\DeclareMathOperator{\coker}{coker}
\DeclareMathOperator{\Hom}{Hom}
\DeclareMathOperator{\Tor}{Tor}
\DeclareMathOperator{\Mor}{Mor}
\DeclareMathOperator{\Kom}{Kom}
\DeclareMathOperator{\sgn}{sgn}
\DeclareMathOperator{\Ext}{Ext}
\DeclareMathOperator{\Ob}{Ob}
\DeclareMathOperator{\Cone}{C}
\DeclareMathOperator{\Cyl}{Cyl}
\DeclareMathOperator{\Der}{D}
\DeclareMathOperator{\K}{K}
\newcommand{\Set}{\mathrm{Set}}
\newcommand{\Top}{\mathrm{Top}}
\newcommand{\Rmod}{\mathrm{R-mod}}
\newcommand{\id}{\mathrm{id}}
\newcommand{\A}{\mathcal{A}}
\newcommand{\B}{\mathcal{B}}
\newcommand{\C}{\mathcal{C}}
\newcommand{\D}{\mathcal{D}}
\newcommand{\Q}{\mathcal{Q}}
\newcommand{\qi}{quasi-isomorphism}

\newcommand{\bigslant}[2]{{\left.\raisebox{.2em}{$#1$}\middle/\raisebox{-.2em}{$#2$}\right.}}
\newcommand{\mf}[1]{{\mathfrak{#1}}}
\newcommand{\mb}[1]{{\mathbb{#1}}}
\newcommand{\mc}[1]{{\mathcal{#1}}}
\newcommand{\mr}[1]{{\mathrm{#1}}}


\newcounter{punkt}

\theoremstyle{plain}
\newtheorem{theorem}[punkt]{Theorem}
\newtheorem{lemma}[punkt]{Lemma}
\newtheorem{definition}[punkt]{Definition}
\newtheorem{fact}[punkt]{Fact}
\newtheorem{proposition}[punkt]{Proposition}
\newtheorem{corollary}[punkt]{Corollary}

\theoremstyle{definition}
\newtheorem{remark}[punkt]{Remark}
\newtheorem{example}[punkt]{Example}
\newtheorem{exercise}[punkt]{Exercise}



\markright{Piotr Suwara\hfill Homological algebra II: December 17, 2014\hfill}

\begin{document}
    \begin{theorem}
        Assume $T$ is of degree $\leq k$, 
        $A \in \Ob(\C)$ is of projective dimension $\leq n$,
        then $L_qT(A,n) = 0$ for $q > k(r+n)$.
    \end{theorem}
    
    \begin{lemma}
        Let $T$ be as above and $X \in s\C$ such that
        $(NX)_i = 0$ for $i>m$.
        Then $N(TX)_i = 0$ for $i>km$.
    \end{lemma}
    
    \begin{definition}[suspension]
        $SA = \coker(A \to CA)$,
        or $(SA)_q = A_{q-1}$ and $d^{SA} = - d^A$.
    \end{definition}
    
    \begin{corollary}
        We have an exact sequence
        $0 \to A \xrightarrow \kappa CA \xrightarrow \pi SA \to 0$.
    \end{corollary}
    
    \begin{definition}
        Let $X \in s\C$. Define {\em cone and suspension} of $X$ by the formulas
        $CX = KCNX, SX = KSNX$.
    \end{definition}
    
    \begin{remark}
        We have an exact sequence (exact on each level)
        $0 \to X \xrightarrow \kappa CX \xrightarrow \pi SX \to 0$.
        
        Applying $T$ we get (not necessarily exact)
        $0 \to TX \xrightarrow{T(\kappa)} T(CX) \xrightarrow{T(\pi)}T(SX) \to 0$.
    \end{remark}
    
    \begin{remark}
        Let $A \xrightarrow f B \xrightarrow g C$ be a~sequence in $C_\ast(\C)$
        such that $g \circ f = 0$ and $B$ is contractible,
        i.e. we have $s_q:B_q \to B_{q+1}$ such that $d^B s + sd^B = \id$.
        Then $gsf:A \to C$ gives a~chain map $SA \to C$
        and hence a~map $H_q(A) \to H_{q+1}(C)$.
    \end{remark}
    
    \begin{theorem}
        $H(gsf)$ does not depend on the choice of $s$.
    \end{theorem}
    
    \begin{definition}[suspension homomorphism]
        The map $\sigma:H_q(TX) \to H_{q+1}(TSX)$ induced by $\kappa$
        and $\pi$ is called a~{\em suspension homomorphism}.
    \end{definition}
    
    \begin{proposition}
        $\sigma$ defines a~natural transformation of functors.
    \end{proposition}
    
    \begin{proposition}
        Assume $T$ additive, then $0 \to T(X) \to T(CX) \to T(SX) \to 0$ exact
        and we have a~long exact sequence of homology groups:
        $\ldots \to 0 \to H_{q+1}(TSX) \to H_q(TX) \to 0 \to \ldots$,
        and $\sigma$ is the inverse of the map in the middle.
    \end{proposition}
    
    \begin{definition}
        Let $T_p^d(A) = T_p(A,\ldots,A)$ ($d$ means diagonal).
    \end{definition}
    
    \begin{definition}
        Define $d_i = \rho \circ T(\alpha_i') \circ \lambda:
        T_p^d(A) \to T_{p-1}^d(A),$ \\
        where $\lambda$ monomorphism $T_p^d(A) \to T(A \oplus \ldots \oplus A)$,
        $\rho$ epimorphism $T(A\oplus \ldots \oplus A) \to T_{p-1}^d(A)$,
        and $d_j':\bigoplus_{i=1}^p A \to \bigoplus_{i=1}^{p-1} A$,
        equal to $(\id, \ldots, \id, (\id + \id)_j, \id,\ldots, \id)$.
    \end{definition}
    
    \begin{definition}
        Let $X \in s\C$. Define a~sequence of simplicial objects in $\C'$:
        $$\mc T  X = (T_1^d(X) \xleftarrow{\partial'} T_2^d(X) 
        \xleftarrow{} T_3^d(X) \xleftarrow{} \ldots,
        \qquad \partial' = \sum_{i=1}^{p-1} (-1)^i d_i.$$
    \end{definition}
    
    \begin{remark}
        $\partial' \circ \partial' = 0.$
    \end{remark}
    
    \begin{corollary}
        Therefore $\mc T  X$ gives a~bicomplex $$(\mc T  X)_{p,q} = T_p^d(X_q)$$
        with horizontal differentials $\partial'$ and vertical differentials 
        from $kX$.
    \end{corollary}
    
    \begin{proposition}
        We have an embedding $i:kTX = (\mc T  X)_{1,\ast} 
            \hookrightarrow \mr{Tot}(\mc T  X)$
        and it is a~chain map of degree $1$.
    \end{proposition}
    
    \begin{theorem}
        There is a~natural isomorphism 
        $\omega: H tot(\mc T  S X) \simeq H(TSX)$
        such that for any $q$ the diagram commutes:
        \begin{tikzpicture}
            \matrix(m)[matrix of math nodes, row sep=1em]
            { & & H_{q+1}(TSX) \\
            H_q TX & &  \\
                & & H_q(\mc T X) \\ };
            \path[->,font=\scriptsize]
            (m-2-1) edge node[auto] {$\sigma$} (m-1-3)
            (m-2-1) edge node[auto] {$i$} (m-3-3)
            (m-3-3) edge node[auto] {$\omega$} (m-1-3);
        \end{tikzpicture}
    \end{theorem}
    
    \begin{definition}[bar construction]
        $\mc T X$ is called the {\em bar construction} for $T$.
    \end{definition}
    
    \begin{corollary}
        If $T$ is additive, then $\sigma$ is an isomorphism.
    \end{corollary}
    
    \begin{corollary}
        If $T$ is of degree $2$, then there exists a~morphism $\beta$
        such that the sequence is exact:
        $\ldots \to H_q T_2(X,X) \xrightarrow \alpha H_q(TX)
        \xrightarrow \sigma H_{q+1}(TSX)
        \to H_{q+1} T_2(X,X) \to H_{q+1}(TX) \to \ldots$.
    \end{corollary}
    
    \begin{corollary}
        There exists a~spectral sequence wchich converges to $H_\ast TSX$ 
        and which satisfies
        \begin{itemize}
            \item $E_{pq}'$ is equal to the complex
            $H_q TX \xleftarrow{H_q(\partial')} H_q T_2(X,X)
            \xleftarrow{H_q(\partial')} H_q T_3(X,X,X) \xleftarrow{} \ldots$,
            \item the homomorphism
            $H_qTX = E_{pq}' \to H_{q+1}TSX$ is the same as $\sigma$.
        \end{itemize}
    \end{corollary}
    
    \begin{definition}
        We say that $X \in s\C$ is {\em trivial below $n$}
        if there exists $X' \in s\C$ which is homotopy equivalent to $X$
        and satisfies $X_i' = 0$ for $i<n$.
    \end{definition}
    
    \begin{lemma}
        If $X$ is projective and $H_q(X) = 0$
        for $q<n$, then $X$ is trivial below $n$.
    \end{lemma}
    
    \begin{remark}[digression]
        A~{\em bisimplicial object}
        is $X_{p,q}$ with $X_{p,q} \to X_{r,s}$
        for any $\alpha:[r]\to [p], \beta:[s]\to [q]$,
        which satisft simplicial identities in both directions.
        
        Every bisimplicial object gives us a~bicomplex $kX$.
        
        If $X$ is bisimplicial, then it comes with a~diagonal
        simplicial object $X_{k,k} \xrightarrow{(\alpha,\alpha)} X_{l,l}$
        (where $\alpha:[l]\to[k]$).
    \end{remark}
    
    \begin{theorem}[Eilenberg-Zilber(-Cantier)]
        There is a~chain homotopy equivalence 
        $k(X_{p,p}) = \mr{tot}(k X_{p,q})$.
    \end{theorem}
    
    \begin{remark}
        Observe that $X_{p,p}$ is in degree $p$
        to the left and $p+p$ to the right.
    \end{remark}
    
    \begin{proposition}
        Let $T:\C^l \to \C'$
        be such that $T(\ldots, 0_j, \ldots) = 0$.
        Let, for $j=1, \ldots, l$, $X^j \in s\C$ be trivial below $n_j$.
        Then $T(X^1, \ldots, X^l)$ is trivial below $n_1 + \ldots + n_l = n$
        (therefore $H_q T(X^1, \ldots, X^l) = 0$ for $q < n$).
    \end{proposition}
    
    \begin{corollary}
        If $X$ is trivial below $n$, then the suspension homomorphism
        $\sigma:H_q(TX) \to H_{q+1}(TSX)$ is an isomorphism
        for $q<2n$ and epimorphism for $q = 2n$.
    \end{corollary}
    
    \begin{definition}[stable derived functors]
        $L_{q+n}(T \bullet, n)$ for $n>q$ is called
        the $q$-th stable derived functor of $T$,
        denoted $L_q^s T(\bullet)$.
    \end{definition}

\end{document}
 
 