\documentclass[12pt]{article}
\usepackage{polski}
\usepackage[utf8]{inputenc}
\usepackage[T1]{fontenc}
\usepackage{amsmath}
\usepackage{amsfonts}
\usepackage{fancyhdr}
\usepackage{lastpage}
\usepackage{multirow}
\usepackage{amssymb}
\usepackage{amsthm}
\usepackage{textcomp}
\frenchspacing
\usepackage{fullpage}
\setlength{\headsep}{30pt}
\setlength{\headheight}{12pt}
%\setlength{\voffset}{-30pt}
%\setlength{\textheight}{730pt}
\pagestyle{myheadings}
%\usepackage{kuvio,amscd,diagrams,dcpic,xymatrix,diagxy}
\usepackage{tikz,paralist,mathtools}
\usetikzlibrary{matrix,arrows,decorations}

\newcommand{\sgn}{{\mathrm{sgn}}\,}
\newcommand{\Kom}{\mathrm{Kom}}
\newcommand{\Mor}{\mathrm{Mor}}
\newcommand{\Hom}{\mathrm{Hom}}
\newcommand{\Ext}{\mathrm{Ext}}
\newcommand{\Ob}{\mathrm{Ob}}
\newcommand{\Cone}{\mathrm{C}}
\newcommand{\Cyl}{\mathrm{Cyl}}
\newcommand{\id}{\mathrm{id}}
\newcommand{\A}{\mathcal{A}}
\newcommand{\B}{\mathcal{B}}
\newcommand{\C}{\mathcal{C}}
\newcommand{\D}{\mathcal{D}}
\newcommand{\Der}{\mathrm{D}}
\newcommand{\Q}{\mathcal{Q}}
\newcommand{\K}{\mathrm{K}}
\newcommand{\qi}{quasi-isomorphism}

\newcommand{\bigslant}[2]{{\left.\raisebox{.2em}{$#1$}\middle/\raisebox{-.2em}{$#2$}\right.}}
\newcommand{\mf}[1]{{\mathfrak{#1}}}
\newcommand{\mb}[1]{{\mathbb{#1}}}
\newcommand{\mc}[1]{{\mathcal{#1}}}
\newcommand{\mr}[1]{{\mathrm{#1}}}


\newcounter{punkt}

\theoremstyle{plain}
\newtheorem{theorem}[punkt]{Theorem}
\newtheorem{lemma}[punkt]{Lemma}
\newtheorem{definition}[punkt]{Definition}
\newtheorem{fact}[punkt]{Fact}
\newtheorem{proposition}[punkt]{Proposition}
\newtheorem{corollary}[punkt]{Corollary}

\theoremstyle{definition}
\newtheorem{remark}[punkt]{Remark}
\newtheorem{example}[punkt]{Example}
\newtheorem{exercise}[punkt]{Exercise}



\markright{Piotr Suwara\hfill Homological algebra II: December 10, 2014\hfill}

\begin{document}
    \begin{definition}[cosimplicial object]
        $X:\Delta \to \C$.
        
        Denote the category of {\em cosimplicial objects} in $\C$ as $c\C$.
    \end{definition}
    
    \begin{theorem}[Dold-Kan again]
        $c\C \simeq \text{cochain complexes over }\C$
    \end{theorem}
    
    \begin{remark}
        $c\C = s(\C^{op})$
    \end{remark}
    
    \begin{definition}
        For $X \in s\C$ define $kX \in C_\ast(\C)$,
        $(kX)_n = X_n$, $d=\sum_{i=0}^n (-1)^i d_i$.
    \end{definition}
    
    \begin{theorem}
        The natural embedding $NX \hookrightarrow kX$
        is a~chain homotopy equivalence.
    \end{theorem}
    
    \begin{remark}
        \begin{itemize}
            \item $NX = \bigslant{kX}{DX}$, where 
            $(DX)_n = \bigcup_{i=0}^{n-1} 
            \mr{im}(s_i:X_{n-1} \to X_n)
            = \mr{im}\left(\prod X_{n-1} \xrightarrow{\prod s_i} \to X_n\right)$.
            \item $kX = NX \oplus DX$,
            \item $DX$ is contractible.
        \end{itemize}
    \end{remark}
    
    \begin{remark}
        One can get $NX$  using $(-1)^n d_n$ instead of $d_0$.
    \end{remark}
    
    \begin{remark}
        Observe that if $\tau_n:[n]\to[n], i \mapsto n-i$,
        and $\alpha^\ast = \tau_n \alpha \tau_m$ for
        $\alpha:[m]\to[n]$, then we get an involution of the category $\Delta$,
        $\alpha \to \alpha^\ast$.
        
        Hence we get an involution of $s\C$,
        $X \to X^\ast$, $(X^\ast)_n = X_n$, $X^\ast(\alpha) = X(\alpha^\ast)$,
        $d_i$ goes to $d_{n-i}$.
        
        We can define $N^\ast X = N(X^\ast)$, $K^\ast C = (KC)^\ast$,
        getting $N^\ast K^\ast = NK$, $K^\ast N^\ast = \mr{Id}$.
    \end{remark}
    
    \begin{theorem}
        \begin{enumerate}
            \item $f_1, f_2:X \to Y$ homotopic in $s\C$
            iff $Nf_1, Nf_2$ are chain homotopic in $C_\ast(\C)$,
            \item $\varphi_1, \varphi_2:C \to D$ 
            are chain homotopic in $C_\ast(\C)$
            iff $K\varphi_1, K\varphi_2$ are homotopic in $s\C$.
        \end{enumerate}
    \end{theorem}
    
    \begin{definition}[simplicial resolution]
        Let $T:\C \to \C'$ functor between abelian categories,
        $\C$ has enough projective objects, 
        $A \in \Ob(\C)$ and $n \in \mb N$.
        
        Then a~pair $(X_\bullet, \xi)$ is called 
        a~{\em siplicial resolution of $A$ of degree $n$}
        (simplicial resolution of $(A,n)$)
        if $X_\bullet \in s\C$, $X_i = 0$ for $i<n$,
        $H_j(X) := H_j(kX) = 0$ for $j>n$
        and $\xi:H_n(X) \to A$ is an isomorphism.
        
        If $\forall_i X_i$ is projective, then $X$
        is a~{\em projective resolution of $(A,n)$}.
        
        Usually we will remove $\xi$ from notation 
        and say that $H_n(X) = A$.
    \end{definition}
    
    \begin{remark}
        \begin{enumerate}
            \item If $X_\bullet$ is a~simplicial resolution of $(A,n)$,
            then $NX$ is a~resolution of $A$ shifted up by $n$.
            If $X_\bullet$ is projective, then $NX$ 
            is a~projective resolution.
            \item If $P \in C_\ast(\C)$ is a~projective resolution of 
            $A$ shifted by $n$, then $KP$ is a~simplicial
            projective resolution of $(A,n)$.
            \item If $\alpha:A \to B$ in $\C$ and $X,Y$ are projective
            resolutions of $(A,n)$ and $(B,n)$, then
            there exists a~simplicial morphism
            $f:X \to Y$ which induces $\alpha = H_n(f)$.
            
            Moreover, $f$ is unique up to homotopy.
        \end{enumerate}
    \end{remark}
    
    \begin{definition}[derived functor]
        Fuctor $L_qT(\bullet, n): \C \to \C'$
        defined below is called 
        {\em $q$-th left derived functor of $T$ of degree $n$},
        where $L_qT(\bullet,n)(A) = H_q(T(X))$,
        where $X$ is any simplicial resolution of $A$.
    \end{definition}
    
    \begin{remark}
        If $T$ is additive, then $k(T(X)) = T(kX)$,
        so $L_qT(A,n) = L_{q-n}T(A)$
        ($L_{q-n}$ from ordinary homotopy category).
    \end{remark}
    
    \begin{remark}
        When $T$ is not additive, then 
        $\sum_{i=0}^n (-1)^i T(d_i)$ is usually not equal
        $T(\sum (-1)^i d_i)$,
        so $k(TX)$ and $T(kX)$ may have different homology.
    \end{remark}
    
    Let $T:\C \to \C'$ functor of abelian categories.
    Assume $T(0) = 0$ (if $T(0)= A$, then take $T' = \ker(T \to T(0) = A)$).
    
    \begin{definition}[cross effect]
        For any ${k \in \mb N}$ we define the 
        {\em $k$-th cross-effect of $T$}
        as a~functor $T_k:\C^k \to \C'$
        such that we get a~functorial decomposition
        $T(A_1 \oplus \ldots \oplus A_k)
        = \bigoplus_{i=1}^k T(A_i)
        \oplus \bigoplus_{i_1<i_2} T_2(A_{i_1}, A_{i_2})
        \oplus \ldots \oplus T_k(A_1, \ldots, A_k)$.
        
        We can define $T_k$ inductively,
        \begin{itemize}
            \item $T_1 = T$,
            \item $T_2(A_1,A_2) = 
            \ker(T(A_1 \oplus A_2) \to T(A_1) \oplus T(A_2))$,
            \item $\ldots$,
            \item $T_k(A_1, \ldots, A_k)
            = \ker(T(A_1 \oplus \ldots \oplus A_n) \xrightarrow{\prod_i}
            \prod_i T(A_1 \oplus \ldots \oplus \hat{A}_i \oplus \ldots \oplus A_k))$.
        \end{itemize}
    \end{definition}
    
    \begin{definition}[functor degree]
        We say that $T$ is of degree $\leq k$ if $T_{k+1} = 0$.
        
        We say that $T$ is of degree $k$ 
        if $T$ is of degree $\leq k$ and $T_k \neq 0$.
    \end{definition}
    
    \begin{theorem}[properties of cross-effects]
        \begin{itemize}
            \item If for some $i$, $A_i = 0$, then $T_k(A_1, \ldots, A_n) = 0$.
            \item $T_k$ is symmetric in its variables.
            \item If we define $s^{(1)}(A) = T_2(A,A_2)$,
            $s^{(2)}(A) = T_2(A_1,A)$,
            then $s_2^{(1)}(A_1, A_2; A_3) = s_2^{(2)}(A_1; A_2, A_3) 
            = T_3(A_1,A_2,A_3)$.
        \end{itemize}
    \end{theorem}
    
    \begin{example}
        $\deg T \leq 1$ iff $T$ is additive.
    \end{example}
    
    \begin{example}
        $T(A) = A^{\otimes 2}$,
        then $T_2(A,B) = (A \otimes B) \oplus (B \otimes A)$
        and it is linear in $A,B$, so $T$ is of degree $2$.
    \end{example}




    
    















\end{document}
 
 