\documentclass[12pt]{article}
\usepackage{polski}
\usepackage[utf8]{inputenc}
\usepackage[T1]{fontenc}
\usepackage{amsmath}
\usepackage{amsfonts}
\usepackage{fancyhdr}
\usepackage{lastpage}
\usepackage{multirow}
\usepackage{amssymb}
\usepackage{amsthm}
\usepackage{textcomp}
\frenchspacing
\usepackage{fullpage}
\setlength{\headsep}{30pt}
\setlength{\headheight}{12pt}
%\setlength{\voffset}{-30pt}
%\setlength{\textheight}{730pt}
\pagestyle{myheadings}
%\usepackage{kuvio,amscd,diagrams,dcpic,xymatrix,diagxy}
\usepackage{tikz,paralist,mathtools}
\usetikzlibrary{matrix,arrows,decorations}

\DeclareMathOperator{\coker}{coker}
\DeclareMathOperator{\Hom}{Hom}
\DeclareMathOperator{\Tor}{Tor}
\DeclareMathOperator{\Mor}{Mor}
\DeclareMathOperator{\Kom}{Kom}
\DeclareMathOperator{\sgn}{sgn}
\DeclareMathOperator{\Ext}{Ext}
\DeclareMathOperator{\Ob}{Ob}
\DeclareMathOperator{\Cone}{C}
\DeclareMathOperator{\Cyl}{Cyl}
\DeclareMathOperator{\Der}{D}
\DeclareMathOperator{\K}{K}
\newcommand{\Set}{\mathrm{Set}}
\newcommand{\Top}{\mathrm{Top}}
\newcommand{\Rmod}{\mathrm{R-mod}}
\newcommand{\id}{\mathrm{id}}
\newcommand{\A}{\mathcal{A}}
\newcommand{\B}{\mathcal{B}}
\newcommand{\C}{\mathcal{C}}
\newcommand{\D}{\mathcal{D}}
\newcommand{\Q}{\mathcal{Q}}
\newcommand{\qi}{quasi-isomorphism}

\newcommand{\bigslant}[2]{{\left.\raisebox{.2em}{$#1$}\middle/\raisebox{-.2em}{$#2$}\right.}}
\newcommand{\mf}[1]{{\mathfrak{#1}}}
\newcommand{\mb}[1]{{\mathbb{#1}}}
\newcommand{\mc}[1]{{\mathcal{#1}}}
\newcommand{\mr}[1]{{\mathrm{#1}}}


\newcounter{punkt}

\theoremstyle{plain}
\newtheorem{theorem}[punkt]{Theorem}
\newtheorem{lemma}[punkt]{Lemma}
\newtheorem{definition}[punkt]{Definition}
\newtheorem{fact}[punkt]{Fact}
\newtheorem{proposition}[punkt]{Proposition}
\newtheorem{corollary}[punkt]{Corollary}

\theoremstyle{definition}
\newtheorem{remark}[punkt]{Remark}
\newtheorem{example}[punkt]{Example}
\newtheorem{exercise}[punkt]{Exercise}



\markright{Piotr Suwara\hfill Homological algebra II: November 19, 2014\hfill}

\begin{document}
    
    \begin{theorem}
        The core $\A = \C^{\geq 0} \cap \C^{\leq 0} \subset \C$ 
        is an abelian category.
    \end{theorem}
    
    \begin{definition}[cohomology object]
        The $i$-th {\em cohomology object} of $X \in \C$
        is defined as
        $$H^0(X) = \tau_{[0,0]}(X) \in \A,$$
        $$H^i(X) = H^0(X[i]) \in \A.$$
    \end{definition}
    
    \begin{definition}[nondegenerate t-structure]
        A~t-structure on $\C$ is {\em nondegenerate} if 
        $\bigcap_n \Ob \C^{\geq n}
        = \bigcap_n \Ob \C^{\leq n} = \{0\}$.
    \end{definition}

    
    \begin{theorem}
        $H^0$ is a~cohomological functor.
        
        If additionally the t-structure is nondegenerate,
        then
        \begin{itemize}
            \item $f:X \to Y$ in $\C$ is an isomorphism 
            iff $\forall_i H^i(f)$ is an isomorphism,
            \item $\Ob(\C^{\leq n}) = 
            \{ X \in \Ob \C: \forall_{i>n} H^i(X) = 0 \}$,
            \item $\Ob(\C^{\geq n}) =
            \{ X \in \Ob \C: \forall_{i<n} H^i(X) = 0 \}$.
        \end{itemize}
    \end{theorem}
    
    \begin{definition}[bounded t-structure]
        A~t-structure is {\em bounded} if it is nondegenerate
        and for any $X \in \C$,
        $H^i(X) \neq 0$ only for a~finite number of $i$.
    \end{definition}
    
    \begin{definition}[$\Ext$]
        $\Ext_\C^i(X,Y) = \Hom_\C(X,Y[i])$
    \end{definition}
    
    \begin{definition}[multiplication on $\Ext$]
        Notice that $\Hom_\C(X,Y[i]) = \Hom_\C(X[k],Y[i+k])$
        and define multiplication 
        $\Ext_\C^i(X,Y) \times \Ext_\C^j(Y,Z) \to 
        \Ext_\C^{i+j}(X,Z)$
        in the most obvious way.
    \end{definition}
    
    \begin{theorem}
        Let $\A$ be a~core of a~bounded t-structure
        on $\C$. Assume $F: \Der^b(\A) \to \C$
        satisfies
        $$F(\Der^b(\A)^{\geq 0}) \subset \C^{\geq 0},$$
        $$F(\Der^b(\A)^{\leq 0}) \subset \C^{\leq 0},$$
        then $F$ is an equivalence of categories 
        iff $\Ext_\C$ is generated by $\Ext_\C^1$ 
        under Yoneda multiplication.
    \end{theorem}
    
    \begin{remark}
        In chain complexes
        $\Ext_\C^i(X,Y):
        0 \to Y \to E_i \xrightarrow{d^i}
        \ldots \xrightarrow{d^2} E_1
        \xrightarrow{d^1} X
        \to 0$.
    \end{remark}
    
    \begin{theorem}
        Assume that $\C$ satisfies additionally:
        \begin{itemize}
            \item TR5. Arbitrary coproducts and products exist in $\C$.
            \item There is a~generating set $\Lambda$
            of objects in $\C$, i.e. set $\Lambda$ such that
            \begin{itemize}
                \item $T(\Lambda) \subset \Lambda$, $T$ -- translation functor,
                \item if $X \in \C$ and 
                $\forall_{\lambda \in \Lambda} \Hom(\lambda,X) = 0$,
                then $X \simeq 0$.
            \end{itemize}
        \end{itemize}
        Then any homological functor
        $H:\C \to \A$ (where $\A$ is abelian)
        which sends coproducts to products is representable,
        i.e. $H = \C(\cdot, h)$.
    \end{theorem}

\end{document}
 
 
