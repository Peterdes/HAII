\documentclass[12pt]{article}
\usepackage{polski}
\usepackage[utf8]{inputenc}
\usepackage[T1]{fontenc}
\usepackage{amsmath}
\usepackage{amsfonts}
\usepackage{fancyhdr}
\usepackage{lastpage}
\usepackage{multirow}
\usepackage{amssymb}
\usepackage{amsthm}
\usepackage{textcomp}
\frenchspacing
\usepackage{fullpage}
\setlength{\headsep}{30pt}
\setlength{\headheight}{12pt}
%\setlength{\voffset}{-30pt}
%\setlength{\textheight}{730pt}
\pagestyle{myheadings}
%\usepackage{kuvio,amscd,diagrams,dcpic,xymatrix,diagxy}
\usepackage{tikz,paralist,mathtools}
\usetikzlibrary{matrix,arrows,decorations}

\DeclareMathOperator{\coker}{coker}
\DeclareMathOperator{\Hom}{Hom}
\DeclareMathOperator{\Tor}{Tor}
\DeclareMathOperator{\Mor}{Mor}
\DeclareMathOperator{\Kom}{Kom}
\DeclareMathOperator{\sgn}{sgn}
\DeclareMathOperator{\Ext}{Ext}
\DeclareMathOperator{\Ob}{Ob}
\DeclareMathOperator{\Cone}{C}
\DeclareMathOperator{\Cyl}{Cyl}
\DeclareMathOperator{\Der}{D}
\DeclareMathOperator{\K}{K}
\newcommand{\Set}{\mathrm{Set}}
\newcommand{\Top}{\mathrm{Top}}
\newcommand{\Rmod}{\mathrm{R-mod}}
\newcommand{\id}{\mathrm{id}}
\newcommand{\A}{\mathcal{A}}
\newcommand{\B}{\mathcal{B}}
\newcommand{\C}{\mathcal{C}}
\newcommand{\D}{\mathcal{D}}
\newcommand{\Q}{\mathcal{Q}}
\newcommand{\qi}{quasi-isomorphism}

\newcommand{\bigslant}[2]{{\left.\raisebox{.2em}{$#1$}\middle/\raisebox{-.2em}{$#2$}\right.}}
\newcommand{\mf}[1]{{\mathfrak{#1}}}
\newcommand{\mb}[1]{{\mathbb{#1}}}
\newcommand{\mc}[1]{{\mathcal{#1}}}
\newcommand{\mr}[1]{{\mathrm{#1}}}


\newcounter{punkt}

\theoremstyle{plain}
\newtheorem{theorem}[punkt]{Theorem}
\newtheorem{lemma}[punkt]{Lemma}
\newtheorem{definition}[punkt]{Definition}
\newtheorem{fact}[punkt]{Fact}
\newtheorem{proposition}[punkt]{Proposition}
\newtheorem{corollary}[punkt]{Corollary}

\theoremstyle{definition}
\newtheorem{remark}[punkt]{Remark}
\newtheorem{example}[punkt]{Example}
\newtheorem{exercise}[punkt]{Exercise}



\markright{Piotr Suwara\hfill Homological algebra II: 15 September 2014\hfill}

\begin{document}
	\begin{definition}
		$X$ is an $H^0$-complex if $H^i(X) \neq 0 \implies i = 0$.
	\end{definition}
	
	\begin{theorem}
		The precomposition of the localization functor $\Q : \Kom(\C) \to \Der(\C)$
		with embedding $i_0: \C \to \Kom(\C)$ defines an equivalence between $\C$
		and the full subcategory of $\Der(\C)$ consisting of $H^0$-complexes.
	\end{theorem}
	
	\begin{definition}
		$X[i] = T^i([X])$ for $X \in \C$.
	\end{definition}

	
	\begin{definition}
		$\C$ -- abelian, then $\Ext_\C^i(X,Y) = \Hom_{\Der(\C)}(X[0], Y[i])$.
	\end{definition}
	
	\begin{remark}
		One does not need projectives or injectives in this definition.
	\end{remark}
	
	\begin{remark}
		$\Ext_\C^i(X,Y) = \Hom_{\Der(\C)}(X[k], Y[k+i])$ for any $k \in \mb{Z}$.
	\end{remark}
	
	\begin{definition}[multiplication]
		There is a~multiplication 
		$$\Ext_\C^i(X,Y) \times \Ext_\C^j(Y,Z) \to \Ext_\C^{i+j}(X,Z)$$
		via composition
		$\Hom_{\Der(\C)}(X[0],Y[i]) \times \Hom_{\Der(\C)}(Y[i],Z[i+j])
		\to \Hom_{\Der(\C)}(X[0],Z[i+j])$.
	\end{definition}
	
	\begin{fact}
		For an exact sequence $0 \to Y' \to Y \to Y'' \to 0$ there is 
		an exact sequence
		$$\ldots\to \Ext^i(X,Y') \to \Ext^i(X,Y) \to \Ext^i(X,Y'') 
		\to \Ext^{i+1}(X,Y') \to \ldots$$
	\end{fact}
	
	\begin{exercise}
		Show that if $X \to Y \to Z \to X[1]$ is distinguished in $\Der(\C)$,
		then we have an exact sequence of abelian groups
		$$\ldots \to \Hom_{\Der(\C)}(U, X[i])
		\to \Hom_{\Der(\C)}(U, Y[i])
		\to \Hom_{\Der(\C)}(U, Z[i])
		\to \Hom_{\Der(\C)}(U, X[i+1]) \to \ldots$$
	\end{exercise}
	
	\begin{theorem}
		$\Ext_\C^0(X,Y) = \Hom_\C(X,Y)$
	\end{theorem}
	
	\begin{theorem}
		$\Ext_\C^i(X,Y) = 0$ for $i<0$.
	\end{theorem}
	
	\begin{theorem}
		Every element in $\Ext_\C^i(X,Y)$ has a~presentation 
		$X[0] \xleftarrow{s} K \xrightarrow{f} Y[i]$,
		where $K_j = 0$ for $j < -i$ and for $j > 0$,
		$K_{-i} = Y$, $f_i = \mr{id}$, and $s$ is a~\qi.
		
		In other words, every such element comes from
		an exact sequence
		$$0 \to Y= K^{-i} \to K^{-i+1} \to K^{-i+2} \to \ldots
		\to K^1 \to K^0 \to X \to 0.$$
	\end{theorem}












\end{document}
 
 
